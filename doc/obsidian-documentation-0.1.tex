\documentclass[a4paper]{article}
\usepackage{multicol}
\usepackage{url}
\usepackage{graphics}

\usepackage{amssymb}
\textheight 9.0in
\columnsep 0.375in
\textwidth 6.0in
\footskip 0.0in
\topmargin -0.0625in
\headheight 0.0in
\headsep 0.0in
\oddsidemargin 0.2in 
\parindent 0pc

\pagestyle{empty}
\newtheorem{sample}{Sample}

\begin{document}
\date{}
\title{\includegraphics{../resources/images/logo_1.png}\textbf{\\}\textbf{\\}\Large \textbf{OBSIDIAN 0.1 DOCUMENTATION
\\}\vspace{0.5cm}specifications}
\author{\textbf{Cl�ment Gam�}\\clement(at)digi-nation(dot)com}

\maketitle
\thispagestyle{empty}

\vspace{3cm}
\begin{figure}[h]
\begin{center}
\label{fig:orga}
\end{center}
\end{figure}

\clearpage

{\textbf{}
{\small\em }}
\vspace{0.5cm}
{\bf\it \\Keywords:}
{\small Peer-to-Peer,Fast R�plication,mirroring,Dexp,Linux }

\tableofcontents

\textbf{\\ \\ \\ \\}

\thispagestyle{plain}


\section{Introduction}

This is the documentation of the Obsidian Peer-to-Peer mirroring Software,
version 0.1. This paper is released under CC-by-SA licence. \\\\In ancient times, People was using a volcanic stone, which amount of 

\section{Overview}

Obsidian is a fast-mirroring, Peer-to-Peer software which was specially designed to maintain large file repositories synchronized with each others, featuring very fast changes propagating. \\\\To achieve this, Obsidian use 2 core technologies: Filesystem Events Listening , and a new protocol specially designed for the occasion: DEXP, acronym for Documents EXchange Protocol. With DEXP, Change Notifications are sent directly to the mirroring peers, so they don't have to poll periodically for file changes.\\\\

To illustrate how fast is Obsidian compared to a classical multiple rsync-based system, we can do the math:\\\\
Let's consider a chain of n peers, each node being synchronized to the next with a periodical rsynch check \[(T_c)\].The time taken by each node to download a change is T_d. In this case, the worst-case propagating time will be:\\\\

\[ T_r = \sum_0^n T_d(n) + T_c(n)\]\\\\
As we can see in this case (rsync) , the more nodes we have, the more time it loses, because of the accumulation the \[T_c\] components.\\\\But in the other case (Obsidian) , the calculation of the maximum propagating time between n nodes is:\\\\
\[T_r  = \sum_0^n T_d(n)\]\\\\
So with obisidan, The more nodes you haves, the more efficient it is compared to a periodic-poll based method.If we now consider a T_c constant for each node, we can evaluate the speed gain with: \[T_g = nT_c\], so yes, the bigger is n, the bigger is the gain.

\section{Global Architecture}

\section{Software Architecture}

\section{Getting Started with Obsidian}

This section describes in detail how to get started with Obsidian, step-by-step and from scratch.


\subsection{prerequisites}

In order to build and run Obsidian, you'll need: \\\\

- A GNU/Linux Op�rating System, with kernel 2.6.19 or higher.\\
- The OpenSSL Library plus its headers (libssl-dev) , 0.9.8 or higher.\\
- Cross-platform make (cmake 2.6 or higher).\\
- The GNU C compiler ( gcc-4.X ).\\\\


\subsection{Fetching the sources}



\subsection{Compilation}

This step explains how to build Obsidian


\subsection{Installation}

To install Obsidian, Nothing is more simple. From the Obsidian Root Directory, just type "make install" , and the Makefile will take care of the rest.

\subsection{Configuration} 

\section{Advanced Usages}

\subsection{Setting up TLS on your obsidian Server}

\subsection{Running multiple sessions of Obsidian on the same server}

In order to circumvence the zero-dir limitation, you can choose to run multiple instances of obsidian at the same time on your server, so your peers can replicate data from the multiple instances.


\end{document}
